\documentclass[twocolumn,amsmath,amssymb, resi]{revtex4}
\pdfoutput=1

\usepackage{natbib}
\usepackage{amsmath}
\usepackage{amssymb}
\usepackage{epsfig}
\usepackage{graphics}
\usepackage{xcolor}

\graphicspath{{figures/}}


%%%%%%%%%%%%%%%%%%%%%%%%%%%%%%
\newcommand{\ii}{\item}
\newcommand{\si}{\subitem}
\newcommand{\bi}{\begin{itemize}}
\newcommand{\ei}{\end{itemize}}
\newcommand{\sub}{\subsection}
\newcommand{\cm}{$\checkmark$}
\newcommand{\lm}{$\boxdot$}
\newcommand{\um}{$\square$}
\newcommand{\todo}[1]{\textcolor{blue}{\textbf [to do: #1]}}
%%%%%%%%%%%%%%%%%%%%%%%%%%%%%%
\begin{document}

\title{Cryogenic Silicon Coating Test Facility}


\author{
 Gautam Venugopalan,
Aaron Markowitz,
Rana X. Adhikari,
Geoffrey Lovelace}
\vspace{1em}
\affiliation{California Institute of Technology}

 
\begin{abstract}

This is a paper about the application of FEA models of thermal noise in mechanical oscillators to the problem of designing a general approach to optimal optomechanical force sensing. (these words are probably nonsense in this order but I think most of the right ones are here)

\end{abstract}
\pacs{??}

\maketitle

\section{Introduction}



\section{Force Sensing}

What is the ideal scale for optomechanical force sensors?

\section{LIGO Suspension Thermal Noise}

We would like to know how the suspension thermal noise scales with test mass size, which will allow us to choose an appropriate scale for various optomechanical add-ons that shape the IFO's quantum noise.

\section{FEA and How It's Done}

The big problem is the strain energy in thin (or microscopic) but lossy subdomains of a macroscopic oscillator. Geoffrey has some code that may be suitable for this, and there are some relatively new techniques for handling these problems with [ellipsoids? what?]

\section{Results}

We have a plot reproducing the aLIGO suspension thermal noise curve, which we can scale with test mass size.

\section{Conclusions and Future Directions}

We will apply this to optimize our ponderomotive squeezing and backaction evasion schemes.

\section{Acknowledgements}



\bibliographystyle{ieeetr}
\bibliography{./bibliography/FEA} 

\end{document}
